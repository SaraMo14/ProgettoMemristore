\documentclass[12pt, a4paper]{book}
\usepackage{amsmath}
\begin{document}

\author{Montese, Pasquini, Terrenzi}

\title{Memristor oscillator on Matlab}
\maketitle

\section{Introduction}
\subsection{What's a memristor?}


Memristor is a passive two terminal electronic device described by a non linear
constitutive relation

                     $v=M(q)i$ or $i=W(\phi)v$

 between the device terminal voltave v and 
terminale current i.

The memeristor constitutive relation is represented by the memristance and the
memductance, defined by this equations: 

$M(q) = \frac{d\phi(q)}{dq} $ and

$ W(fi)=\frac{dq(\phi)}{d\phi}$

In particular we note that a memristor is passive if, and only if, its 
memristance is non-negative 

$M(q)=\frac{d\phi(q)}{dq}\geq 0$

We assume that the memristor ic characterized by the "monotone-increasing" and
"piecewise-linear" nonlinearity 

$fi(q)=bq+0.5(a-b)(|q+1|-|q-1|)$ and 

$q(\phi)=d\phi+0.5(c-d)(|\phi+1|-|\phi-1|)$

 Consequently, the memristance and the memductance are defined by 
 
\begin{equation}
M(q)=\frac{d\phi(q)}{dq}=
\begin{cases}
a, |q|<1
\\
b, |q|>1
\end{cases}
\end{equation}

\subsection{History}

Leon Chua postulated the idea of memristor in 1971, but the first real memristor
was build only in 2008 by R. Stanley Williams from the Hewlett-Packard Company.
The device instantly generated interest all over the world for its potential
applications, for example in non-volatible memories, in next-generation computers,
but also in cryptography.

\subsection{Our project}

The main theme of our project is the Matlab implementation of all the plots that
can see in the paper.
We studied the functions of various types of chaotic oscillators and finded a way
to represent their signal on a plot.
So, mainly, we implemented this functions on Matlab, solved differential 
equations and created plots.

But we'll see the details of our work in the next pages.

\section{Plots, equations, code}
\subsection{Fig.10}
Fourth-order canonical memristor oscillator:
\begin{equation}
\begin{cases}
\frac{dx}{dt}=\alpha(y-W(\omega)x)
\\
\frac{dy}{dt}=z-x
\\
\frac{dz}{dt}=-y\beta+z\gamma
\\
\frac{d\omega}{dt}=x
\\
\end{cases}
\end{equation}

con le seguenti equazioni:

\begin{equation}
q(\omega)=b\omega+0.5(a-b)(|\omega+1|-|\omega-1|)
\end{equation}
\begin{equation}
W(\omega)=\frac{dq(\omega)}{d\omega}=
\begin{cases}
a, |\omega|<1
\\
b, |\omega|>1
\end{cases}
\end{equation}

\subsection{Fig.12}
Fourth-order canonical memristor oscillator:
\begin{equation}
\begin{cases}
\frac{dx}{dt}=\alpha(y-W(\omega)x)
\\
\frac{dy}{dt}=-\xi(x+z)
\\
\frac{dz}{dt}=y\beta
\\
\frac{d\omega}{dt}=x
\\
\end{cases}
\end{equation}

con le seguenti equazioni:

\begin{equation}
q(\omega)=b\omega+0.5(a-b)(|\omega+1|-|\omega-1|)
\end{equation}
\begin{equation}
W(\omega)=\frac{dq(\omega)}{d\omega}=
\begin{cases}
a, |\omega|<1
\\
b, |\omega|>1
\end{cases}
\end{equation}

\subsection{Fig.17}
Third-order canonical memristor oscillator:
\begin{equation}
\begin{cases}
\frac{dx}{dt}=\alpha(y-W(z)x)
\\
\frac{dy}{dt}=-\xi x+\beta y
\\
\frac{dz}{dt}=x
\\
\end{cases}
\end{equation}

con le seguenti equazioni:

\begin{equation}
q(z)=bz+0.5(a-b)(|z+1|-|z-1|)
\end{equation}
\begin{equation}
W(z)=\frac{dq(z)}{dz}=
\begin{cases}
a, |z|<1
\\
b, |z|>1
\end{cases}
\end{equation}

\subsection{Fig.23}
Second-order canonical memristor oscillator:
\begin{equation}
\begin{cases}
\frac{dx}{dt}=\alpha(\beta-W(y)x)
\\
\frac{dy}{dt}=x
\\
\end{cases}
\end{equation}

con le seguenti equazioni:

\begin{equation}
q(y)=by+0.5(a-b)(|y+1|-|y-1|)
\end{equation}
\begin{equation}
W(y)=\frac{dq(y)}{dt}=
\begin{cases}
a, |y|<1
\\
b, |y|>1
\end{cases}
\end{equation}

\subsection{Fig.26}
Fourth-order canonical memristor oscillator:
\begin{equation}
\begin{cases}
\frac{dx}{dt}=\alpha(y-x+\xi x(\omega)x)
\\
\frac{dy}{dt}=x-y+z
\\
\frac{dz}{dt}=-y\beta-z\gamma
\\
\frac{d\omega}{dt}=x
\\
\end{cases}
\end{equation}

con le seguenti equazioni:

\begin{equation}
q(\omega)=b\omega+0.5(a-b)(|\omega+1|-|\omega-1|)
\end{equation}
\begin{equation}
W(\omega)=\frac{dq(\omega)}{dt}=
\begin{cases}
a, |\omega|<1
\\
b, |\omega|>1
\end{cases}
\end{equation}

\subsection{Fig.29}
Third-order canonical memristor oscillator:
\begin{equation}
\begin{cases}
\frac{dx}{dt}=\alpha(y-W(z)x+\gamma x)
\\
\frac{dy}{dt}=\beta x
\\
\frac{dz}{dt}=x
\\
\end{cases}
\end{equation}

con le seguenti equazioni:

\begin{equation}
q(z)=bz+0.5(a-b)(|z+1|-|z-1|)
\end{equation}
\begin{equation}
W(z)=\frac{dq(z)}{dz}=
\begin{cases}
a, |z|<1
\\
b, |z|>1
\end{cases}
\end{equation}

\subsection{Fig.32 and Fig.33}
Second-order canonical memristor oscillator:
\begin{equation}
\begin{cases}
y=(\gamma-W(z))x
\\
\frac{dy}{dt}=\beta x
\\
\frac{dz}{dt}=x
\\
\end{cases}
\end{equation}

con le seguenti equazioni:

\begin{equation}
q(z)=bz+0.5(a-b)(|z+1|-|z-1|)
\end{equation}
\begin{equation}
W(z)=\frac{dq(z)}{dz}=
\begin{cases}
a, |z|<1
\\
b, |z|>1
\end{cases}
\end{equation}

\subsection{Fig.35 and Fig.36}
First-order canonical memristor oscillator:
\begin{equation}
\begin{cases}
\frac{dx}{dt}=\frac{e}{W(x)-\beta}
\end{cases}
\end{equation}

con le seguenti equazioni:

\begin{equation}
q(x)=bx+0.5(a-b)(|x+1|-|x-1|)
\end{equation}
\begin{equation}
W(x)=\frac{dq(x)}{dx}=
\begin{cases}
a, |x|<1
\\
b, |x|>1
\end{cases}
\end{equation}

\end{document}